\documentclass[letter, 11pt]{book}

\usepackage[letter]{geometry}
\usepackage[english]{babel}
\usepackage{graphicx}
\usepackage{wrapfig}
\usepackage{enumitem}
\usepackage{fancyhdr}
\usepackage{amssymb, amsmath}
\usepackage{index}
\usepackage{tcolorbox}

\graphicspath{{figures/}}

\tcbuselibrary{theorems}
\newtcbtheorem[number within=section]{theorem}{Theorem}%
{colback=green!5,colframe=green!35!black,fonttitle=\bfseries}{th}
\newtcbtheorem[number within=section]{definition}{Definition}%
{colframe=gray!50!black,fonttitle=\bfseries}{}
\begin{document}
\title{\Large{\textbf{Differential Equations Student Notebook}}}
\author{Julian Alexander Alvarez}
\date{February 2, 2020}
\maketitle
\let\cleardoublepage\clearpage
\tableofcontents

\setcounter{page}{2}
\fancyhf{}
\renewcommand{\headrulewidth}{2pt}
\renewcommand{\footrulewidth}{1pt}

\fancyhead[LE]{\leftmark}
\fancyhead[RO]{\leftmark}
\fancyfoot[LE,RO]{\thepage}


\chapter{Preface}
Welcome to my \textit{Differential Equations Student Notebook}. I created this ``borderline textbook'' notebook
as a way for me and others to study and/or learn Differential Equations from. You may be asking: ``What qualifications do you have to be making this \textit{Differential Equations Notebook}?''
and my response to that is that I am not. \textbf{I AM NOT QUALIFIED}. I am just a student who thought that making a textbook would assist me in learning about DEs. However, I have taken Calculus I, II, III, Discrete Mathematics, Linear Algerbra, and any other foundation-building mathematical class that University of Houston - Clear Lake provides (if that makes you trust me more).


It would be greatly appreciated if you simply took this as a study resource for any current, soon-to-be-current, already-took-this-and-just-forgot student.
Therefore, without further ado, lets begin our dive into DIFFERENTIAL EQUATIONS.
\newpage
\chapter{The Very Beginning}
\section{Requirements}
As it is in every area of mathematics, there is a certain level of knowledge that is to be expected to be known by students.
Requirements for \textit{Differential Equations} is prior experience in Calculus I and II.

Of course, it is \textbf{expected that you are a BORDERLINE SAVANT in Calculus}. If you are not absolutely comfortable with Calculus, I reccommend that you REALLY go over it.
\section{Types of Differential Equations}
There are two types of Differential Equations: Ordinary Differential Equations (ODEs) and Partial Differential Equations (PDEs).
The two differences between a \textbf{ODE} and a \textbf{PDE} is that an ODE is (as the name suggests) an equation that contains ordinary differentials. An example of an ODE is as follows:
\begin{equation}\label{Ex. 1}
\centered
\frac{dy}{dx} = 4y - 2x
\end{equation}
An example of a PDE is as follows:
\begin{equation}\label{Ex. 2}
\centered
\frac{\partial y}{\partial x} = 17x^2 -3y*\sin{y}
\end{equation}
Did you spot the difference? The differentiating factor between example 2.1 and 2.2 is $\frac{dy}{dx}$ and $\frac{\partial y}{\partial x}$. \textit{One contains partials while the other does not}.
\begin{definition}{Differential Equation}{}
	An equation that contains the derivatives of one or more unknown functions or variables ($f(x), y$, etc.), with respect to one or more independent variables is a \textbf{DIFFERENTIAL EQUATION}.
\end{definition}

\section{Classification By Order}
	The ordering of a differential equation is set by these three tenets:
\begin{itemize}
	\item Type
	\item Order
	\item Linearity
\end{itemize}

\end{document}
